\documentclass{article}
% basics
\usepackage{amsfonts}
\usepackage{enumitem}
\usepackage{float}
\usepackage{graphicx}

% unique math expressions:
\usepackage{amsmath}
\DeclareMathOperator*{\andloop}{\wedge}
\DeclareMathOperator*{\pr}{Pr}
\DeclareMathOperator*{\approach}{\longrightarrow}
\DeclareMathOperator*{\eq}{=}

% grey paper
\usepackage{xcolor}
\pagecolor[rgb]{0.11,0.11,0.11}
\color{white}

% embedded code sections
\usepackage{listings}
\definecolor{codegreen}{rgb}{0,0.6,0}
\definecolor{codegray}{rgb}{0.5,0.5,0.5}
\definecolor{codepurple}{rgb}{0.58,0,0.82}
\lstdefinestyle{mystyle}{
    commentstyle=\color{codegreen},
    keywordstyle=\color{magenta},
    numberstyle=\tiny\color{codegray},
    stringstyle=\color{codepurple},
    basicstyle=\ttfamily\footnotesize,
    breakatwhitespace=false,         
    breaklines=true,                 
    captionpos=b,                    
    keepspaces=true,                 
    numbers=left,                    
    numbersep=5pt,                  
    showspaces=false,                
    showstringspaces=false,
    showtabs=false,                  
    tabsize=2
}

\lstset{style=mystyle}

\begin{document}
\author{Yosef Goren - Or Rubin}
\title{Graph Seperation - Final Report}
\maketitle
\part*{Abstract}

\part*{Introduction}
A famous result in the theory of message passing networks (MPNs) states
that their expressive power is equal to the 1-WL isomorphism test.
More accurately, MPNs is a general framework for graph neural networks that can be
instatiated in many ways. While any realization of an MPN cannot separate
graphs unless they are separable by the WL isomorphism test, the converse
claim can be showed to be true for specific MPN constructions. The two clas-
sical constructions of MPNs which are equivalent to WL are by Xu [2] and by
Morris [1].

In this project we will focus on the construction by Xu and a disadvantage
of this construction. A follow up project (e.g., for thesis) would aim to fix these
disadvantage, for example by using a sorting based MPN construction.


A key result for the proof of separation of GINs in Lemma 5. The lemma
states that if $\mathcal{X}$ is a countable set and N is a fixed natural number, there
A key result for the proof of separation of GINs in Lemma 5. The lemma
states that if $\mathcal{X}$ is a countable set and N is a fixed natural number, there
exists a function $f : \mathcal{X} \rightarrow R$ such that, for every $\mathcal{X}1, \mathcal{X}2 \subseteq \mathcal{X}$
of size < N ,exists a function $f : \mathcal{X} \rightarrow R$ such that, for every
$\mathcal{X}1, \mathcal{X}2 \subseteq \mathcal{X}$ of size $\leq N$.

Since there are a finite amount of linear segments to it,
and the input range is infinite, there must be at-least
one segment which is infinitly long: in particular,
there must be a segment $[x_1,x_1+4], x_1\in\mathbb{Z}$
where $m$ corresponds to some affine function.\\
If we denote $y_1=m(x_1), y_2=m(x_1+4), s = \frac{y_2-y_1}{4}$.\\
We can find that affine function as:
\[
    \forall x\in \{x_1+i\mid i\in[3]\}, m(x)=s\cdot x+y_1
\]
Denote $X_1=\{x_1,x_1+4\}, X_2=\{x_1+1,x_1+3\}$.
Now we can see that:
\[
    \sum_{x\in X_1}m(x)=
    (2x_1+4)s+2y_1=
    \sum_{x\in X_2}m(x)
\]


\part*{Methods}
\section*{Metrics}
To determine the quality of seperation yeilded by the model, we have considered a few metrics.\\
Note that we are using a finite size of $\mathcal{X}$ in these tests. Hence, $\mathcal{X}$
is also a finite set in the definitions of these metrics (the metrics are a function of both the model and the test sample).
\begin{itemize}
    \item Given a model $M:\mathcal{X}\rightarrow\mathbb{R}$, 
    we will look at $M^{-1}:\mathbb{R}\rightarrow\mathcal{X}$ s.t. 
    $\forall r\in\mathbb{R}, M^{-1}(r)=\{x\in\mathcal{X}:M(x)=r\}$.

    The first metric is calculated as follows:
    \[
        \frac{\sum_{r\in\mathbb{R}\wedge |M^{-1}(r)|>1}|M^{-1}(r)|}
        {|\mathcal{X}|}
    \]

    This metric is the propotion between the number of elements 
    that are collided and the total number of elements in the test set.
    
    We will refer to this as the 'count metric'.

    \item The second metric is the probability of collision
    between two uniformly sampled samples from the test set.
    This metric can be calculated as follows:

    \[
        \frac{\sum_{r\in\mathbb{R}\wedge M^{-1}(r)\neq\emptyset}|M^{-1}(r)|\cdot\left(|M^{-1}(r)|-1\right)}
        {|\mathcal{X}|\cdot\left(|\mathcal{X}|-1\right)}
    \]

    We will refer to this metric as the 'probability metric'.


    \section*{Training}
    Since we aim to prevert collisions, an intuative huristic at achiving this would be to keep
    outputs as distant as possible.
    Hence, we suggest the loss to be the negative distance between the outputs of the model:
    \[
        L(G_1,G_2)=-|M(G_1)-M(G_2)|    
    \]

    In order to train our GIN model, we followed the Xu [2].


\end{itemize}

\part*{Experiments}

\part*{Conclusion}

\part*{Future Work}

\part*{References}
[1] Morris, D. (2019). Message Passing Neural Networks. arXiv preprint arXiv:1910.00825.
[2] Xu, K., Li, W., \& Leskovec, J. (2019). How Powerful are Graph Neural Networks?. arXiv preprint arXiv:1810.00826.

\end{document}