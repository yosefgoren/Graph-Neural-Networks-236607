\documentclass{article}
% basics
\usepackage{amsfonts}
\usepackage{enumitem}
\usepackage{float}
\usepackage{graphicx}

\usepackage{hyperref}

% unique math expressions:
\usepackage{amsmath}
\DeclareMathOperator*{\andloop}{\wedge}
\DeclareMathOperator*{\pr}{Pr}
\DeclareMathOperator*{\approach}{\longrightarrow}
\DeclareMathOperator*{\eq}{=}
\DeclareMathOperator*{\aprx}{\sim}

% grey paper
\usepackage{xcolor}
\pagecolor[rgb]{0.11,0.11,0.11}
\color{white}

% embedded code sections
\usepackage{listings}
\definecolor{codegreen}{rgb}{0,0.6,0}
\definecolor{codegray}{rgb}{0.5,0.5,0.5}
\definecolor{codepurple}{rgb}{0.58,0,0.82}
\lstdefinestyle{mystyle}{
    commentstyle=\color{codegreen},
    keywordstyle=\color{magenta},
    numberstyle=\tiny\color{codegray},
    stringstyle=\color{codepurple},
    basicstyle=\ttfamily\footnotesize,
    breakatwhitespace=false,         
    breaklines=true,                 
    captionpos=b,                    
    keepspaces=true,                 
    numbers=left,                    
    numbersep=5pt,                  
    showspaces=false,                
    showstringspaces=false,
    showtabs=false,                  
    tabsize=2
}

\lstset{style=mystyle}

\begin{document}
\author{Yosef Goren}
\title{Graph Neural Network}
\maketitle
\setcounter{section}{1}
\section{}
\begin{enumerate}[label=(\alph*)]
    % \addtocounter{enumi}{1}
    \item This is a private case of (b).
    \item Let $m$ be some piecewise linear function.\\
        Since there are a finite amount of linear segments to it,
        and the input range is infinite, there must be at-least
        one segment which is infinitly long: in particular,
        there must be a segment $[x_1,x_1+4], x_1\in\mathbb{Z}$
        where $m$ corresponds to some affine function.\\
        If we denote $y_1=m(x_1), y_2=m(x_1+4), s = \frac{y_2-y_1}{4}$.\\
        We can find that affine function as:
        \[
            \forall x\in \{x_1+i\mid i\in[3]\}, m(x)=s\cdot x+y_1
        \]
        Denote $X_1=\{x_1,x_1+4\}, X_2=\{x_1+1,x_1+3\}$.
        Now we can see that:
        \[
            \sum_{x\in X_1}m(x)=
            (2x_1+4)s+2y_1=
            \sum_{x\in X_2}m(x)
        \]
\end{enumerate}


\end{document}